\documentclass[11pt]{article}
\usepackage[margin=1in]{geometry}
\usepackage{times}
\usepackage{multicol}
\usepackage{titlesec}
\usepackage{enumitem}
\usepackage{setspace}
\usepackage{titling}
\usepackage{authblk}
\usepackage{fancyhdr}

% Custom section formatting
\titleformat{\section}{\large\bfseries}{\thesection}{1em}{}
\titleformat{\subsection}{\normalsize\bfseries}{\thesubsection}{1em}{}
\titleformat{\subsubsection}{\normalsize\itshape}{\thesubsubsection}{1em}{}

% Remove numbering from sections
\setcounter{secnumdepth}{0}

\pagestyle{fancy}
\fancyhf{}
\rhead{CyCon 2026}
\lhead{NATO CCDCOE}
\cfoot{\thepage}

\title{This Is the Title}
\author{}
\date{}

\begin{document}

\maketitle

\vspace{-2em}
\begin{multicols}{2}
\noindent
\textbf{John P. Smith} \\
Assistant Professor \\
School of Law \\
Discovery University \\
Jamestown, VA, United States \\
john.smith@discovery.edu

\vfill\null
\columnbreak

\noindent
\textbf{Arthur King*} \\
Senior Lecturer \\
Department of Cyber Security \\
University of Camelot \\
Avalon, United Kingdom \\
aking@camelot.ac.uk
\end{multicols}

\vspace{1em}

\noindent\textbf{Abstract:} This is the article abstract; it should be about 200 to 300 words long. It summarizes the article, outlining the basic structure, main hypothesis and conclusions.

\vspace{1em}

\noindent\textbf{Keywords:} three to six keywords or short keyword phrases, non-capitalized, separated with commas

\section{Introduction}
This is the text of the introduction. The paper is to be written in Times New Roman, 11 pt, single-spaced, formatted in a single column, no paragraph indentation.

\section{Second Section}
Further sections follow this format. If you need to structure the text further, you can use second and third-level headings.

\subsection{This Is the Second-Level Heading Format}
Please do not use numerical numbering. Instead, label these headings with letters or leave them unnumbered.

\subsubsection{This Is the Third-Level Heading Format}
Use these headings if you need to further sub-divide your article.

\section{Third Section}
This is yet another section of your article. When referring to your text, refer to Sections, not to page numbers.

\begin{quote}
Quotations of three or more lines should be put in a separate paragraph, introduced usually by a colon: The paragraph is then indented on both sides, adjusted to a block format and centred, without quotation marks.
\end{quote}

\section{Conclusion}
And this is the conclusion.

\section*{Acknowledgements}
Here you can, but do not have to, say thank you to your colleagues, family or your pet animal. It is not necessary to thank the CyCon reviewers for their comments.

\end{document}

