% !TEX TS-program = xelatex
\documentclass[aspectratio=169]{beamer}
\usetheme[numbering=fraction,progressbar=head]{metropolis}
\setbeamersize{text margin left=8mm,text margin right=8mm}

% --- CUSTOM FOOTER WITH TLP MARKING ---
\setbeamertemplate{frame footer}{\textbf{TLP:CLEAR} – Information may be shared freely without restriction.}
\setbeamercolor{frame footer}{fg=cyber-accent}

% --- FONTS (XeLaTeX) ---
\usepackage{fontspec}
\setsansfont[
  Ligatures=TeX, Scale=1.02,
  BoldFont = {IBM Plex Sans Bold},
  ItalicFont = {IBM Plex Sans Italic}
]{IBM Plex Sans}
\setmonofont{IBM Plex Mono}

% --- ICONS ---
\usepackage{fontawesome5}

% --- COLORS (cyber palette) ---
\definecolor{cyber-bg}{HTML}{0B0F14}
\definecolor{cyber-ink}{HTML}{E6EEF5}
\definecolor{cyber-accent}{HTML}{217EAA}
\definecolor{cyber-mid}{HTML}{7D9CB7}
\definecolor{cyber-soft}{HTML}{8CA4AC}
\setbeamercolor{normal text}{fg=cyber-ink,bg=cyber-bg}
\setbeamercolor{frametitle}{fg=cyber-ink,bg=cyber-bg}
\setbeamercolor{progress bar}{fg=cyber-accent,bg=cyber-soft}
\setbeamercolor{alerted text}{fg=cyber-accent}

% --- LINKS ---
\hypersetup{colorlinks=true, linkcolor=cyber-accent, urlcolor=cyber-accent}

\title{Range\#42:\\Building Europe’s Open Cyber Range}
\author{NC3 Luxembourg \& DIGISQUAD}
\institute{NCC-LU project}
\date{2025}

\begin{document}

%--------------------------------------------------
\begin{frame}
  \titlepage
\end{frame}
%--------------------------------------------------

\begin{frame}{Range\#42: Building Europe’s Open Cyber Range}
\textbf{Mission:}\\[4pt]
Develop an open-source, modular, and automated cyber-range platform that enables hands-on cyber-defence training, simulation, and experimentation — fully deployable within national or EU infrastructures.\\[8pt]

\textbf{Strategic relevance:}
\begin{itemize}
  \item Reinforces European cyber resilience and digital sovereignty.
  \item Aligned with NIS2, Cyber Solidarity Act, and Digital Europe Programme priorities.
  \item Bridges research, education, and operational defence across member states.
\end{itemize}

\textbf{Partners:}
\begin{itemize}
  \item \textbf{NC3 Luxembourg} – coordination and standardisation leadership.
  \item \textbf{DIGISQUAD} – core platform engineering and automation.
  \item Co-funded under \textbf{NCC-LU-S (101127115)}.
\end{itemize}
\end{frame}
%--------------------------------------------------

\begin{frame}{Achievements in 2025}
\textbf{Technical \& organisational progress:}
\begin{itemize}
  \item End-to-end deployment automation via \textbf{Ansible + Proxmox}.
  \item Secure zero-trust networking through \textbf{Tailscale / WireGuard}.
  \item Integrated observability with \textbf{Wazuh} SIEM.
  \item Early \textbf{visual lab-designer prototype} to democratise scenario creation.
\end{itemize}

\textbf{Knowledge \& innovation contributions:}
\begin{itemize}
  \item 100+ vulnerabilities and misconfigurations documented (20 deployable).
  \item Stable architecture validated on Linux-based clusters.
  \item \textbf{Research dissemination:} CyCon 2026 paper on \textit{AI automation for metric gathering, community building, standardisation, and open-source ethics.}
\end{itemize}

\textbf{AI focus:}
\begin{itemize}
  \item AI for scenario classification, telemetry correlation, and exercise-metrics automation.
  \item Foundation for AI-assisted red/blue teaming and autonomous range orchestration.
\end{itemize}
\end{frame}
%--------------------------------------------------

\begin{frame}{The Road Ahead (2026–2027)}
\textbf{Next priorities:}
\begin{itemize}
  \item Broaden scenario library: ICS/OT, AI-driven threat emulation, cross-border exercises.
  \item Finalise user interfaces for orchestration and training management.
  \item Support multi-network, distributed deployments for EU-wide cooperation.
\end{itemize}

\textbf{Strategic development path:}
\begin{itemize}
  \item Embed AI modules for adaptive difficulty and real-time exercise evaluation.
  \item Formalise engagement with the ECCC / NCC ecosystem for interoperability.
  \item Establish a European open repository for cyber-training standards and assets.
\end{itemize}

\textbf{Call to action:}\\[4pt]
\textit{Support Range 42 to anchor Europe’s sovereign, intelligent, and interoperable cyber-training ecosystem.}
\end{frame}
%--------------------------------------------------

\end{document}
