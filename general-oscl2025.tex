% !TEX TS-program = xelatex
\documentclass[aspectratio=169]{beamer}

% --- THEME & LAYOUT ---
\usetheme[numbering=fraction,progressbar=head]{metropolis} % modern, minimal  [oai_citation:5‡tug.org](https://tug.org/docs/latex/beamertheme-metropolis/metropolistheme.pdf?utm_source=chatgpt.com)
\setbeamersize{text margin left=8mm,text margin right=8mm}

% --- FONTS (XeLaTeX) ---
\usepackage{fontspec}
\setsansfont[
  Ligatures=TeX, Scale=1.02,
  BoldFont = {IBM Plex Sans Bold},
  ItalicFont = {IBM Plex Sans Italic}
]{IBM Plex Sans} % CTAN 'plex' package  [oai_citation:6‡CTAN](https://ctan.org/tex-archive/fonts/plex?utm_source=chatgpt.com)
\setmonofont{IBM Plex Mono}

% --- ICONS ---
\usepackage{fontawesome5} % Font Awesome 5 icons  [oai_citation:7‡CTAN](https://ctan.org/pkg/fontawesome5?utm_source=chatgpt.com)

% --- COLORS (cyber palette) ---
\definecolor{cyber-bg}{HTML}{0B0F14}   % charcoal
\definecolor{cyber-ink}{HTML}{E6EEF5}  % offwhite
\definecolor{cyber-accent}{HTML}{217EAA}% blue accent (trust)  [oai_citation:8‡Color Hex](https://www.color-hex.com/color-palette/113704?utm_source=chatgpt.com)
\definecolor{cyber-mid}{HTML}{7D9CB7}  % secondary
\definecolor{cyber-soft}{HTML}{8CA4AC} % tertiary
\setbeamercolor{normal text}{fg=cyber-ink,bg=cyber-bg}
\setbeamercolor{frametitle}{fg=cyber-ink,bg=cyber-bg}
\setbeamercolor{progress bar}{fg=cyber-accent,bg=cyber-soft}
\setbeamercolor{alerted text}{fg=cyber-accent}

% --- LINKS ---
\hypersetup{colorlinks=true, linkcolor=cyber-accent, urlcolor=cyber-accent}

% --- CODE / BOXES ---
\usepackage{listings}
\lstset{
  basicstyle=\ttfamily\footnotesize,
  backgroundcolor=\color{black!5},
  frame=single, rulecolor=\color{cyber-soft},
  keywordstyle=\color{cyber-accent}\bfseries,
  commentstyle=\color{cyber-soft},
  showstringspaces=false
}
\usepackage[most]{tcolorbox}
\tcbset{colback=black!5!cyber-bg, colframe=cyber-soft, coltext=cyber-ink, boxsep=1mm, left=2mm, right=2mm}

% --- SPEAKER NOTES ---
\usepackage{pgfpages}
% Toggle one of the following as needed during rehearsal/presenting:
% \setbeameroption{hide notes}         % Slides only
% \setbeameroption{show notes}         % Notes on second screen (presenter)
% \setbeameroption{show only notes}    % Notes only
% See references for Beamer notes usage.  [oai_citation:9‡TeX - LaTeX Stack Exchange](https://tex.stackexchange.com/questions/483764/add-notes-to-latex-beamer-and-view-them?utm_source=chatgpt.com)

% --- TITLE INFO ---
%\title{\faShieldAlt\; Range42 — Semi-Technical Overview}
\title{\faIcon{shield-alt}\; Range42 — Semi-Technical Overview}
\subtitle{Modular Cyber Range Platform for Real-World Readiness}
\author{NC3 / Range42 Team}
\date{\today}
\institute{\faServer\; Proxmox \quad \faCogs\; Ansible \quad \faProjectDiagram\; Orchestration \quad \faBinoculars\; Telemetry}

\begin{document}

% --- TITLE ---
\begin{frame}
  \titlepage
  \note[item]{Goal: 20-minute semi-technical tour of Range42’s building blocks and roadmap.}
  \note[item]{Audience: technical staff who aren’t deep specialists.}
\end{frame}

% --- AGENDA ---
\begin{frame}{Agenda}
  \begin{enumerate}
    \item What Range42 is \& why it matters
    \item Architecture at a glance
    \item Repositories by function (highlights)
    \item Risks, governance, and quality gates
    \item 90-day roadmap \& demo plan
  \end{enumerate}
  \note[item]{Keep flow brisk; ~2–3 min per major section.}
\end{frame}

% --- WHAT IS RANGE42 ---
\begin{frame}{What is Range42?}
  \begin{itemize}
    \item \textbf{Modular cyber range platform} for offensive, defensive, and hybrid training.
    \item \textbf{Reproducible IaC}: build, deploy, document labs via Proxmox, Ansible, Docker.
    \item \textbf{Private APIs} for orchestration \& telemetry; developer toolkits for pipelines.
  \end{itemize}
  \begin{tcolorbox}
    \faInfoCircle\; Built to simulate \emph{real-world incidents} safely, with isolation, snapshots, and telemetry.
  \end{tcolorbox}
  \note[item]{Anchor on “safe realism”: isolation, resets, data trails.}
\end{frame}

% --- ARCH OVERVIEW ---
\begin{frame}{Architecture at a Glance}
  \begin{itemize}
    \item \textbf{Hypervisor layer}: Proxmox VMs/LXCs; snapshots; network segments.
    \item \textbf{Automation layer}: Ansible roles orchestrate lifecycle, network, firewall, images.
    \item \textbf{Control plane}: Backend API (routes for VM/Net/Runner); Kong gateway.
    \item \textbf{UX}: Deployer UI (visual design), EMP mockup (exercise mgmt), trainee access.
    \item \textbf{Observability}: Wazuh for logs/alerts; structured telemetry.
  \end{itemize}
  \note[item]{Emphasize modularity: builder/deployer/runner + gateway.}
\end{frame}

% === REPOSITORY HIGHLIGHTS (use your JSON) ===

\section{Automation}

\begin{frame}{range42-ansible\_roles-private-devkit \; \faTerminal}
  \textbf{Status:} Active \hfill \textbf{Lang:} Shell\\[2mm]
  \textbf{Purpose:} Helper scripts for Proxmox/Ansible ops (VM/LXC, firewall, snapshots, storage).\\[2mm]
  \textbf{Highlights}
  \begin{itemize}
    \item DevKit scripts standardize Proxmox ops \& JSON/text transformations.
    \item \alert{Zero Bandit findings}; add CI and security baseline.
    \item Prioritize documentation and license finalization.
  \end{itemize}
  \note[item]{Mature and actively used to drive orchestration; governance \& CI next. Tighten contributor UX (docs, editorconfig, SECURITY).}
\end{frame}

\begin{frame}{range42-ansible\_roles-proxmox\_controller \; \faCogs}
  \textbf{Status:} Active \hfill \textbf{Lang:} Ansible/YAML\\[2mm]
  \textbf{Purpose:} Manage VM/LXC lifecycle, networking, storage, firewall, snapshots, templates, ISOs.\\[2mm]
  \textbf{Next Steps}
  \begin{itemize}
    \item Add CI with \texttt{ansible-lint} and idempotence tests.
    \item Document variables and example playbooks.
  \end{itemize}
  \note[item]{Breadth is strong; reliability needs CI and smoke tests. Improve onboarding with examples and var docs.}
\end{frame}

\section{Control Plane}

\begin{frame}{range42-backend-api \; \faProjectDiagram}
  \textbf{Status:} Active \hfill \textbf{Lang:} Python/Shell/YAML\\[2mm]
  \textbf{Purpose:} Routes for Proxmox control, runner bundles, admin/debug; schemas \& curl helpers.\\[2mm]
  \textbf{Security/Quality}
  \begin{itemize}
    \item Scans clean (\texttt{bandit}, \texttt{pip-audit}/safety).
    \item \alert{Add CI and unit tests}; finalize license; add \texttt{SECURITY} policy.
  \end{itemize}
  \note[item]{Good structure; production hardening: CI, tests, dependency pinning, runtime config validation.}
\end{frame}

\begin{frame}{range42-api-definitions \; \faSitemap}
  \textbf{Status:} Stale \hfill \textbf{Lang:} —\\[2mm]
  \textbf{Purpose:} Placeholder for API definitions/schemas.\\[2mm]
  \textbf{Decision Point}
  \begin{itemize}
    \item Repo is empty: \textbf{archive or seed OpenAPI specs}.
    \item If kept: minimal README \& roadmap.
  \end{itemize}
  \note[item]{Reduce noise or make it the single source of API truth.}
\end{frame}


% Insert this after the "What is Range42?" frame and before "Architecture at a Glance"

% --- COMPETITIVE LANDSCAPE ---
\begin{frame}{Range42 vs. Other Cyber Ranges}
  \begin{table}
    \scriptsize
    \begin{tabular}{l|cccc}
      \textbf{Feature} & \textbf{Range42} & \textbf{Commercial SaaS} & \textbf{Cloud Native} & \textbf{Traditional} \\
      \hline
      \alert{Open Architecture} & \checkmark & \texttimes & \texttimes & \texttimes \\
      \alert{IaC/GitOps} & \checkmark & \textasciitilde & \checkmark & \texttimes \\
      Private Deployment & \checkmark & \texttimes & \textasciitilde & \checkmark \\
      Cost Control & \checkmark & \texttimes & \textasciitilde & \checkmark \\
      Full Data Custody & \checkmark & \texttimes & \texttimes & \checkmark \\
      API Orchestration & \checkmark & \checkmark & \checkmark & \texttimes \\
      Rapid Reset/Snapshots & \checkmark & \checkmark & \textasciitilde & \textasciitilde \\
      Custom Scenarios & \checkmark & \texttimes & \textasciitilde & \checkmark \\
    \end{tabular}
  \end{table}
  \vspace{2mm}
  \begin{tcolorbox}
    \faLightbulb\; \textbf{Range42's Edge}: Full control, reproducibility, and cost-effectiveness without vendor lock-in.
  \end{tcolorbox}
  \note[item]{Emphasize open architecture and IaC as key differentiators.}
  \note[item]{Commercial SaaS: SimSpace, Immersive Labs, RangeForce, Cyberbit.}
  \note[item]{Cloud Native: AWS/Azure ranges, cloud-specific solutions.}
  \note[item]{Traditional: Physical labs, legacy virtualized platforms.}
\end{frame}

% Alternative: Feature comparison with more detail
\begin{frame}{Why Range42? Key Differentiators}
  \begin{columns}[t]
    \column{0.5\textwidth}
    \textbf{vs. Commercial SaaS} \\[2mm]
    \begin{itemize}
      \item \alert{Full data custody} (no external telemetry)
      \item \alert{No per-seat licensing} (fixed infra cost)
      \item \alert{Unlimited customization} (scenarios as code)
      \item Deploy on-premises or private cloud
    \end{itemize}
    
    \column{0.5\textwidth}
    \textbf{vs. Traditional Ranges} \\[2mm]
    \begin{itemize}
      \item \alert{Reproducible IaC} (version-controlled)
      \item \alert{API-driven automation} (no manual setup)
      \item \alert{Fast reset cycles} (snapshots in seconds)
      \item Modern observability stack
    \end{itemize}
  \end{columns}
  \vspace{4mm}
  \begin{tcolorbox}
    \faCheckCircle\; \textbf{Best of Both Worlds}: Enterprise automation + full control + transparent costs.
  \end{tcolorbox}
  \note[item]{Highlight hybrid approach: modern automation without vendor lock-in.}
\end{frame}

\section{UX Layers}

\begin{frame}{range42-deployer-ui \; \faDrawPolygon}
  \textbf{Status:} Prototype \hfill \textbf{Lang:} Vue/TS\\[2mm]
  \textbf{Purpose:} VueFlow-based visual orchestrator to design, validate, deploy infra.\\[2mm]
  \textbf{Next Steps}
  \begin{itemize}
    \item Unit/e2e tests exist; wire CI.
    \item Stabilize API contracts and error handling.
  \end{itemize}
  \note[item]{Great UX direction; add CI and contract tests against backend; export/import hardening.}
\end{frame}

\begin{frame}{range42-emp-mockup \; \faChalkboardTeacher}
  \textbf{Status:} Prototype \hfill \textbf{Lang:} Vue/TS\\[2mm]
  \textbf{Purpose:} Exercise Management Platform scaffold (routing, unit tests, basic pages).\\[2mm]
  \textbf{Next Steps}
  \begin{itemize}
    \item Resolve 2 low npm advisories.
    \item Define MVP and wire CI.
  \end{itemize}
  \note[item]{Seed project; define MVP/architecture first. Add CI (lint/test) \& dependency automation.}
\end{frame}

\section{Infrastructure \& Content}

\begin{frame}{range42-infrastructures-installers \; \faCubes}
  \textbf{Status:} Active \hfill \textbf{Lang:} Ansible/Shell\\[2mm]
  \textbf{Purpose:} Bundles \& demo labs to build base VM templates, admin/student infra, service stacks.\\[2mm]
  \textbf{Quality Gates}
  \begin{itemize}
    \item Add \texttt{ansible-lint} \& Molecule idempotence tests in CI.
    \item Document secrets/vault \& inventories.
  \end{itemize}
  \note[item]{High impact infra content; enforce idempotence; provide reference inventories (redacted).}
\end{frame}

\begin{frame}{range42-inventory \; \faBoxes}
  \textbf{Status:} Active \hfill \textbf{Lang:} Ansible/YAML\\[2mm]
  \textbf{Purpose:} Vulnerable scenarios \& admin utilities; software stacks \& training assets.\\[2mm]
  \textbf{Next Steps}
  \begin{itemize}
    \item Introduce scenario taxonomy \& tags.
    \item Add CI for lint/sanity-check of roles.
  \end{itemize}
  \note[item]{Cornerstone repo: standardize metadata \& compatibility matrices; add dry-run checks to prevent drift.}
\end{frame}

\section{Documentation}

\begin{frame}{range42-documentation-private-obsidian \; \faBook}
  \textbf{Status:} Active \hfill \textbf{Lang:} Markdown/Shell\\[2mm]
  \textbf{Purpose:} Private Obsidian vault (notes, canvases, API drafts, tooling).\\[2mm]
  \textbf{Actions}
  \begin{itemize}
    \item Create export pipeline to public docs.
    \item Track decisions \& version architecture canvases.
  \end{itemize}
  \note[item]{Add cadence for pruning; consider diagrams-as-code for reviews; plan public extraction.}
\end{frame}

% --- RISKS / GOVERNANCE ---
\section{Risks \& Governance}

\begin{frame}{Key Risks \& Gaps \; \faExclamationTriangle}
  \begin{itemize}
    \item \alert{No CI/CD in most repos}: enable lint, tests, security scans, release checks.
    \item \alert{License placeholders}: finalize SPDX headers, LICENSE files.
    \item \alert{No SECURITY policy}: define vuln disclosure and triage path.
    \item Docs/editor config gaps: contributor friction.
  \end{itemize}
  \note[item]{Make CI the gate: ansible-lint, molecule, pytest, npm audit, bandit, safety/pip-audit.}
\end{frame}

% Additional slides for Range42 presentation - GitHub Repository Cloner
% Insert these slides after the "Risks & Governance" section

\section{Repository Management}

\begin{frame}{GitHub Organization Repository Cloner}
  \faDownload\;
  \textbf{Status:} Production Ready \hfill \textbf{Lang:} Bash/Shell\\[2mm]
  \textbf{Purpose:} Mass clone and audit all Range42 repositories with comprehensive standards compliance checking.\\[2mm]
  \textbf{Key Features}
  \begin{itemize}
    \item \alert{Mass Repository Cloning}: Clone all accessible org repositories automatically.
    \item \alert{Smart Organization}: Separates public/private repos into dedicated directories.
    \item \alert{Comprehensive Audits}: Standards compliance checking with line-by-line output.
  \end{itemize}
  \note[item]{Production tool actively used for Range42 org management; reduces manual overhead.}
\end{frame}

\begin{frame}{Repository Standards Compliance}
  \faClipboardCheck\;
  \textbf{Automated Sanity Checks}
  \begin{itemize}
    \item \textbf{License Validation}: LICENSE files with template placeholder detection.
    \item \textbf{Documentation}: README, CHANGELOG, CONTRIBUTING, SECURITY policies.
    \item \textbf{Git Configuration}: .gitignore, .editorconfig standards.
    \item \textbf{CI/CD Detection}: GitHub Actions, GitLab CI, Jenkins, Travis, etc.
    \item \textbf{Templates}: Issue/PR templates, CODE OF CONDUCT.
  \end{itemize}
  \begin{tcolorbox}
    \faInfoCircle\; Addresses governance gaps: \emph{license placeholders, missing SECURITY policies, CI/CD standardization}.
  \end{tcolorbox}
  \note[item]{Directly addresses the governance risks identified earlier; automated compliance checking.}
\end{frame}

\begin{frame}{Cloner Usage and Benefits}
  \faTerminal\;
  \textbf{Quick Commands} (clone all, audit all, check specific repo)
  \begin{tcolorbox}
    \texttt{./gh\_repo\_cloner.sh}\\[2mm]
    \texttt{./gh\_repo\_cloner.sh --sanity-check}\\[2mm]
    \texttt{./gh\_repo\_cloner.sh -s range42-backend-api}
  \end{tcolorbox}
  \textbf{Governance Benefits}
  \begin{itemize}
    \item Batch identification of repos missing LICENSE/SECURITY files.
    \item \alert{CI/CD detection} across all repositories.
    \item Focused analysis for onboarding and code reviews.
  \end{itemize}
  \note[item]{Practical tool that supports the governance roadmap; measurable compliance metrics.}
\end{frame}

\begin{frame}{Integration with Range42 Roadmap}
  \faRocket\;
  \textbf{Supports 90-Day Goals}
  \begin{itemize}
    \item \textbf{Week 1-3}: Identify repos needing CI foundations via automated audit.
    \item \textbf{Week 2-5}: Batch detection of missing LICENSE and SECURITY policies.
    \item \textbf{Week 4-8}: Track backend-UI contract compliance across repos.
    \item \textbf{Ongoing}: Monthly audits for standards drift prevention.
  \end{itemize}
  \begin{tcolorbox}
    \faLightbulb\; \textbf{Measurable Progress}: Generate compliance dashboards and track improvement over time.
  \end{tcolorbox}
  \note[item]{Tool becomes enabler for roadmap execution; provides metrics for demo milestones.}
\end{frame}

% --- ROADMAP ---
\section{Roadmap}

\begin{frame}{90-Day Roadmap \; \faRoad}
  \begin{enumerate}
    \item \textbf{Week 1–3}: CI foundations (lint, tests, scans) across active repos.
    \item \textbf{Week 2–5}: License \& SECURITY policies; contributor docs.
    \item \textbf{Week 4–8}: Backend–UI contract tests; stabilize API definitions.
    \item \textbf{Week 6–12}: Inventory taxonomy and scenario coverage matrix.
  \end{enumerate}
  \note[item]{Show measurable progress: CI badges, release notes, scenario matrix.}
\end{frame}

% --- DEMO PLAN ---
\begin{frame}{Demo Plan \; \faPlay}
  \begin{itemize}
    \item \textbf{Scope}: Deploy \emph{demo lab} via installers; control via backend routes; visualize in Deployer UI.
    \item \textbf{Success}: Green CI, idempotent runs, reset-to-clean-state, baseline telemetry in Wazuh.
    \item \textbf{Artifacts}: Recorded steps, configs as code, and a short walkthrough video/GIF.
  \end{itemize}
  \note[item]{Keep demo focused: one scenario, full loop (deploy \textrightarrow{} observe \textrightarrow{} reset).}
\end{frame}

% --- WRAP-UP ---
\begin{frame}{Takeaways \; \faCheck}
  \begin{itemize}
    \item Range42 = modular, reproducible cyber range for realistic training.
    \item Strong building blocks; \alert{governance \& CI} are the multiplier.
    \item Clear 90-day path to production-grade readiness.
  \end{itemize}
  \note[item]{Invite questions; point to repos and docs roadmap.}
\end{frame}

\end{document}
