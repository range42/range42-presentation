% !TEX TS-program = xelatex
\documentclass[aspectratio=169]{beamer}

% --- THEME & LAYOUT ---
\usetheme[numbering=fraction,progressbar=head]{metropolis}
\setbeamersize{text margin left=8mm,text margin right=8mm}

% --- CUSTOM FOOTER WITH TLP MARKING ---
\setbeamertemplate{frame footer}{\textbf{TLP:CLEAR} – Information may be shared freely without restriction.}
\setbeamercolor{frame footer}{fg=cyber-accent}

% --- FONTS (XeLaTeX) ---
\usepackage{fontspec}
\setsansfont[
  Ligatures=TeX, Scale=1.02,
  BoldFont = {IBM Plex Sans Bold},
  ItalicFont = {IBM Plex Sans Italic}
]{IBM Plex Sans}
\setmonofont{IBM Plex Mono}

% --- ICONS ---
\usepackage{fontawesome5}

% --- COLORS (cyber palette) ---
\definecolor{cyber-bg}{HTML}{0B0F14}
\definecolor{cyber-ink}{HTML}{E6EEF5}
\definecolor{cyber-accent}{HTML}{217EAA}
\definecolor{cyber-mid}{HTML}{7D9CB7}
\definecolor{cyber-soft}{HTML}{8CA4AC}
\setbeamercolor{normal text}{fg=cyber-ink,bg=cyber-bg}
\setbeamercolor{frametitle}{fg=cyber-ink,bg=cyber-bg}
\setbeamercolor{progress bar}{fg=cyber-accent,bg=cyber-soft}
\setbeamercolor{alerted text}{fg=cyber-accent}

% --- LINKS ---
\hypersetup{colorlinks=true, linkcolor=cyber-accent, urlcolor=cyber-accent}

% --- CODE / BOXES ---
\usepackage{listings}
\lstset{
  basicstyle=\ttfamily\footnotesize,
  backgroundcolor=\color{black!5},
  frame=single, rulecolor=\color{cyber-soft},
  keywordstyle=\color{cyber-accent}\bfseries,
  commentstyle=\color{cyber-soft},
  showstringspaces=false
}
\usepackage[most]{tcolorbox}
\tcbset{colback=black!5!cyber-bg, colframe=cyber-soft, coltext=cyber-ink, boxsep=1mm, left=2mm, right=2mm}

% --- GRAPHICS ---
\usepackage{graphicx}

% --- SPEAKER NOTES ---
\usepackage{pgfpages}
% \setbeameroption{hide notes}
% \setbeameroption{show notes}
% \setbeameroption{show only notes}

% --- TITLE INFO ---
\title{\faIcon{shield-alt}\; Range42 status update – Semi-Technical Overview}
\subtitle{Modular Cyber Range Platform for Real-World Readiness}
\author{NC3 / Range42 Team}
\date{\today}
\institute{\faServer\; Proxmox \quad \faCogs\; Ansible \quad \faProjectDiagram\; Orchestration \quad \faBinoculars\; Telemetry}

\begin{document}

% --- TITLE ---
\begin{frame}
  \titlepage
  \note[item]{Goal: 20-minute semi-technical tour of Range42's building blocks and current status.}
  \note[item]{Audience: technical staff who aren't deep specialists.}
\end{frame}

% Slide for PDF page 1
\begin{frame}[plain]
  \centering
  \includegraphics[
    page=1,
    width=\paperwidth,
    height=\paperheight,
    keepaspectratio
  ]{pdf_slides/LHC_NC3.pdf}
\end{frame}

% Slide for PDF page 2
\begin{frame}[plain]
  \centering
  \includegraphics[
    page=2,
    width=\paperwidth,
    height=\paperheight,
    keepaspectratio
  ]{pdf_slides/LHC_NC3.pdf}
\end{frame}

% --- PUBLIC MONEY PUBLIC CODE ---
\begin{frame}{Public money, public code. Let's go all the way, shall we?}
  \begin{columns}[T]
    \column{0.65\textwidth}
    \textbf{Current Team}
    \begin{itemize}
      \item Core development team of 3 contributors
      \item Mix of InfoSec, DevOps engineers, and platform architects
      \item Collaborative model with NC3 and academic partners
    \end{itemize}
    
    \textbf{Current Funding}
    \begin{itemize}
      \item Public grant for cyber training infrastructure
      \item Open-source model: no licensing fees, transparent development
      \item Investment in reusable, community-tooling
    \end{itemize}
    
    \column{0.3\textwidth}
    \vspace{5mm}
    \begin{figure}
      \centering
      \includegraphics[width=\textwidth]{images/logos/FSFE_Public_Money_Public_Code_logo.pdf}
    \end{figure}
  \end{columns}
  \note[item]{Emphasize public funding → public ownership → public benefit.}
\end{frame}

\begin{frame}{Public money, public code. Let's go all the way, shall we?}
  \begin{columns}[T]
    \column{0.65\textwidth}
    \textbf{Current Challenges}
    \begin{itemize}
      \item Governance standardization across 13+ repositories from the get go
      \item Balancing rapid development with documentation maturity
      \item Recruitment, we need YOU, to apply here →
    \end{itemize}
    
    \column{0.3\textwidth}
    \vspace{10mm}
    \begin{figure}
      \centering
      \includegraphics[width=\textwidth]{images/work-with-us-qrcode.png}
    \end{figure}
  \end{columns}
  \note[item]{Range42 embodies PMPC principles: taxpayer-funded development results in freely available, auditable infrastructure.}
\end{frame}

% --- AGENDA ---
\begin{frame}{Agenda}
  \begin{enumerate}
    \item What Range42 is \& why it matters
    \item Competitive landscape
    \item Architecture at a glance
    \item Repository audit findings (13 repositories analyzed)
    \item Risks, governance, and quality gates
  \end{enumerate}
  \note[item]{Keep flow brisk; ~3–4 min per major section.}
\end{frame}

% --- WHAT IS RANGE42 ---
\begin{frame}{What is Range42?}
  \begin{itemize}
    \item \textbf{Modular cyber range platform} for offensive, defensive, and hybrid training.
    \item \textbf{Reproducible IaC}: build, deploy, document labs via Proxmox, Ansible, Docker.
    \item \textbf{Private APIs} for orchestration \& telemetry; developer toolkits for pipelines.
  \end{itemize}
  \begin{tcolorbox}
    \faInfoCircle\; Built to simulate \emph{real-world incidents} safely, with isolation, snapshots, and telemetry.
  \end{tcolorbox}
  \note[item]{Anchor on "safe realism": isolation, resets, data trails.}
\end{frame}

% --- COMPETITIVE LANDSCAPE ---
\begin{frame}{Range42 vs. Other Cyber Ranges}
  \begin{table}
    \scriptsize
    \begin{tabular}{l|cccc}
      \textbf{Feature} & \textbf{Range42} & \textbf{Commercial SaaS} & \textbf{Cloud Native} & \textbf{Traditional} \\
      \hline
      \alert{Open Architecture} & \checkmark & \texttimes & \texttimes & \texttimes \\
      \alert{IaC/GitOps} & \checkmark & \textasciitilde & \checkmark & \texttimes \\
      Private Deployment & \checkmark & \texttimes & \textasciitilde & \checkmark \\
      Cost Control & \checkmark & \texttimes & \textasciitilde & \checkmark \\
      Full Data Custody & \checkmark & \texttimes & \texttimes & \checkmark \\
      API Orchestration & \checkmark & \checkmark & \checkmark & \texttimes \\
      Rapid Reset/Snapshots & \checkmark & \checkmark & \textasciitilde & \textasciitilde \\
      Custom Scenarios & \checkmark & \texttimes & \textasciitilde & \checkmark \\
    \end{tabular}
  \end{table}
  \vspace{2mm}
  \begin{tcolorbox}
    \faLightbulb\; \textbf{Range42's Edge}: Full control, reproducibility, and cost-effectiveness without vendor lock-in.
  \end{tcolorbox}
  \note[item]{Emphasize open architecture and IaC as key differentiators.}
  \note[item]{Commercial SaaS: SimSpace, Immersive Labs, RangeForce, Cyberbit.}
\end{frame}

% --- ARCH OVERVIEW ---
\begin{frame}{Architecture at a Glance}
  \begin{itemize}
    \item \textbf{Hypervisor layer}: Proxmox VMs/LXCs; snapshots; network segments.
    \item \textbf{Automation layer}: Ansible roles orchestrate lifecycle, network, firewall, images.
    \item \textbf{Control plane}: Backend API (routes for VM/Net/Runner); Kong gateway.
    \item \textbf{UX}: Deployer UI (visual design), EMP mockup (exercise mgmt), trainee access.
    \item \textbf{Observability}: Wazuh for logs/alerts; structured telemetry.
  \end{itemize}
  \note[item]{Emphasize modularity: builder/deployer/runner + gateway.}
\end{frame}

\begin{frame}{Architecture: Logical Components → Repository Mapping}
  \begin{figure}
    \centering
    \includegraphics[width=0.95\textwidth]{images/diagrams/architecture.png}
  \end{figure}
  \vspace{-2mm}
  \begin{tcolorbox}
    \faInfoCircle\; \textbf{Left}: Logical architecture flow. \textbf{Right}: Actual GitHub repository structure.
  \end{tcolorbox}
  \note[item]{Left side shows logical architecture; right side shows actual GitHub repos.}
  \note[item]{Emphasize how web components flow through API gateway to backend, which controls Proxmox via Ansible role.}
  \note[item]{Deployable catalog (playbooks + catalog repos) contains the infrastructure-as-code content.}
\end{frame}

% === REPOSITORY AUDIT OVERVIEW ===
\section{Repository Audit Findings}

\begin{frame}{Organization-Wide Audit Results \; \faClipboardCheck}
  \textbf{13 Repositories Analyzed} (2 public, 11 private)\\[3mm]
  
  \textbf{Critical Findings}
  \begin{itemize}
    \item \alert{9/13 repositories} have LICENSE template placeholders (\texttt{<year>}, \texttt{<name of author>})
    \item \alert{13/13 repositories} missing SECURITY.md vulnerability disclosure policy
    \item \alert{12/13 repositories} have no CI/CD pipeline
    \item \alert{0 repositories} have all required governance files
  \end{itemize}
  \vspace{2mm}
  \begin{tcolorbox}
    \faCheckCircle\; \textbf{Good News}: Zero high-severity security findings across all code scans (bandit, pip-audit, npm audit).
  \end{tcolorbox}
  \note[item]{Sanity check performed 2025-09-30 using gh-repo-organizer tool.}
  \note[item]{Security posture is excellent; governance is the primary gap.}
\end{frame}

% === REPOSITORY MANAGEMENT TOOL ===
\begin{frame}{gh-repo-organizer: Audit Tool \; \faDownload}
  \textbf{Status:} Active \hfill \textbf{Commits:} 18 \hfill \textbf{Lang:} Bash\\[2mm]
  \textbf{Purpose:} Bash script for mass cloning and auditing GitHub organization repositories with comprehensive standards compliance checking.\\[2mm]
  \textbf{Key Features}
  \begin{itemize}
    \item Production-ready tool for mass repository cloning and standards auditing
    \item Detects LICENSE template placeholders, missing documentation, and CI/CD configurations
    \item Zero security findings; needs CI pipeline and contributor documentation
  \end{itemize}
  \note[item]{This tool generated the audit findings you just saw.}
  \note[item]{It's production-ready but needs CI integration and contributor guidelines for full maturity.}
\end{frame}

\begin{frame}{Repository Standards Compliance Checks}
  \faClipboardCheck\;
  \textbf{Automated Sanity Checks Performed}
  \begin{itemize}
    \item \textbf{License Validation}: LICENSE files with template placeholder detection
    \item \textbf{Documentation}: README, CHANGELOG, CONTRIBUTING, SECURITY policies
    \item \textbf{Git Configuration}: .gitignore, .editorconfig standards
    \item \textbf{CI/CD Detection}: GitHub Actions, GitLab CI, Jenkins, Travis, etc.
    \item \textbf{Templates}: Issue/PR templates, CODE OF CONDUCT
  \end{itemize}
  \begin{tcolorbox}
    \faInfoCircle\; Directly addresses governance gaps: \emph{license placeholders, missing SECURITY policies, CI/CD standardization}.
  \end{tcolorbox}
  \note[item]{Tool provides automated compliance checking across the entire organization.}
\end{frame}

% === AUTOMATION LAYER ===
\section{Automation}

\begin{frame}{range42-ansible\_roles-private-devkit \; \faTerminal}
  \textbf{Status:} Active \hfill \textbf{Commits:} 179 \hfill \textbf{Lang:} Shell\\[2mm]
  \textbf{Purpose:} Helper scripts for Proxmox and Ansible operations including VM/LXC management, firewall rules, and JSON transformations.\\[2mm]
  \textbf{Key Findings}
  \begin{itemize}
    \item DevKit provides 100+ helper scripts following strict naming convention
    \item Zero Bandit findings; actively used with 179 commits but lacks CI pipeline
    \item Priority: finalize LICENSE placeholders and add governance documentation
  \end{itemize}
  \note[item]{Mature and heavily used repository with excellent naming conventions.}
  \note[item]{Top priority is adding CI with shellcheck/shfmt and completing governance files.}
\end{frame}

\begin{frame}{range42-ansible\_roles-proxmox\_controller \; \faCogs}
  \textbf{Status:} Active \hfill \textbf{Commits:} 108 \hfill \textbf{Lang:} Ansible/YAML\\[2mm]
  \textbf{Purpose:} Ansible role for managing Proxmox nodes via API: VMs, LXC containers, networking, storage, firewall, and snapshots.\\[2mm]
  \textbf{Key Findings}
  \begin{itemize}
    \item Comprehensive Ansible role managing full Proxmox lifecycle with 108 commits
    \item Broad functionality coverage but lacks CI pipeline for ansible-lint and idempotence tests
    \item Immediate actions: finalize LICENSE, add CI, and document variables with examples
  \end{itemize}
  \note[item]{Core infrastructure role with strong breadth.}
  \note[item]{Needs production hardening through CI with ansible-lint and Molecule.}
\end{frame}

% === CONTROL PLANE ===
\section{Control Plane}

\begin{frame}{range42-backend-api \; \faProjectDiagram}
  \textbf{Status:} Active \hfill \textbf{Commits:} 125 \hfill \textbf{Lang:} Python/Shell/YAML\\[2mm]
  \textbf{Purpose:} FastAPI backend orchestrating Proxmox deployments via Ansible, with routes for VM control, networking, and bundle execution.\\[2mm]
  \textbf{Key Findings}
  \begin{itemize}
    \item FastAPI backend with 125 commits; clean security scans (bandit, pip-audit, safety)
    \item Comprehensive route structure for Proxmox control and bundle orchestration
    \item Production hardening needed: add CI pipeline, unit tests, finalize LICENSE, SECURITY policy
  \end{itemize}
  \note[item]{Well-structured backend with excellent security posture.}
  \note[item]{Priority actions are adding CI with pytest and completing governance documentation.}
\end{frame}

\begin{frame}{range42-api-definitions \; \faSitemap}
  \textbf{Status:} Stale \hfill \textbf{Commits:} 5 \hfill \textbf{Lang:} JSON\\[2mm]
  \textbf{Purpose:} Placeholder repository for OpenAPI/Swagger specifications defining the Range42 backend API contracts.\\[2mm]
  \textbf{Key Findings}
  \begin{itemize}
    \item Placeholder repository with OpenAPI specs but minimal content and only 5 commits
    \item Missing LICENSE file entirely; needs decision on archive versus active development
    \item If kept: seed with comprehensive API definitions and add CI for spec validation
  \end{itemize}
  \note[item]{Decision point: either archive to reduce noise or commit to making it the single source of API truth.}
\end{frame}

\begin{frame}{range42-api-gw \; \faIcon{shield-alt}}
  \textbf{Status:} Active \hfill \textbf{Commits:} 5 \hfill \textbf{Lang:} N/A\\[2mm]
  \textbf{Purpose:} Kong API Gateway configuration for authentication, ACL, and access control policies in front of backend API.\\[2mm]
  \textbf{Key Findings}
  \begin{itemize}
    \item Kong API Gateway repository for authentication and access control with minimal content
    \item LICENSE has template placeholders; needs Kong declarative configs or deployment manifests
    \item Add documentation for Kong setup, plugin configuration, and integration with backend API
  \end{itemize}
  \note[item]{Early-stage repository that needs Kong configuration files and deployment guides.}
  \note[item]{Critical for production security but currently underdeveloped.}
\end{frame}

% === UX LAYERS ===
\section{UX Layers}

\begin{frame}{range42-deployer-ui \; \faDrawPolygon}
  \textbf{Status:} Prototype \hfill \textbf{Commits:} 28 \hfill \textbf{Lang:} Vue/TS\\[2mm]
  \textbf{Purpose:} VueFlow-based visual orchestrator for designing, validating, and deploying Range42 infrastructure through node-based interface.\\[2mm]
  \textbf{Key Findings}
  \begin{itemize}
    \item VueFlow-based UI with node-based infrastructure design; unit/e2e tests exist but no CI
    \item Good UX foundation with i18n support and localStorage data management
    \item Wire CI for automated testing, stabilize API contracts with backend, resolve npm advisory
  \end{itemize}
  \note[item]{Promising prototype with solid testing foundation.}
  \note[item]{Key next steps are enabling CI to run existing tests and hardening API contract validation.}
\end{frame}

\begin{frame}{range42-emp-mockup \; \faChalkboardTeacher}
  \textbf{Status:} Prototype \hfill \textbf{Commits:} 1 \hfill \textbf{Lang:} Vue/TS\\[2mm]
  \textbf{Purpose:} Exercise Management Platform scaffold with basic Vue 3 routing, TypeScript setup, and unit test foundation.\\[2mm]
  \textbf{Key Findings}
  \begin{itemize}
    \item Exercise Management Platform seed project with 1 commit; basic Vue 3 scaffold only
    \item Missing LICENSE entirely; has 2 low npm advisories and no CI pipeline
    \item Define MVP scope and architecture first, then add CI and resolve dependency vulnerabilities
  \end{itemize}
  \note[item]{Very early prototype requiring architectural definition before development continues.}
  \note[item]{Must add LICENSE, resolve npm advisories, and establish CI pipeline once MVP scope is clear.}
\end{frame}

% === INFRASTRUCTURE & CONTENT ===
\section{Infrastructure \& Content}

\begin{frame}{range42-playbooks \; \faCubes}
  \textbf{Status:} Active \hfill \textbf{Commits:} 75 \hfill \textbf{Lang:} Ansible/YAML\\[2mm]
  \textbf{Purpose:} Centralized Ansible playbooks organizing scenarios and bundles for CLI or backend API orchestration of infrastructure deployments.\\[2mm]
  \textbf{Key Findings}
  \begin{itemize}
    \item Central playbook repository with 75 commits organizing scenarios and reusable bundles
    \item Clear structure with test scripts but no CI for ansible-lint or syntax validation
    \item Add CI pipeline, introduce scenario taxonomy with tags, and document compatibility matrices
  \end{itemize}
  \note[item]{Well-organized orchestration layer with good separation between scenarios and bundles.}
  \note[item]{Needs CI enforcement for ansible-lint and comprehensive documentation.}
\end{frame}

\begin{frame}{range42-catalog \; \faBoxes}
  \textbf{Status:} Active \hfill \textbf{Commits:} 91 \hfill \textbf{Lang:} Ansible/YAML\\[2mm]
  \textbf{Purpose:} Collection of reusable Ansible roles and Docker/Compose stacks for deploying vulnerable scenarios and infrastructure bundles.\\[2mm]
  \textbf{Key Findings}
  \begin{itemize}
    \item Catalog of Ansible roles and Docker stacks for deploying training scenarios with 91 commits
    \item Volatile tree structure organizing CVEs and misconfigurations by technology type
    \item Add CI for ansible-lint, introduce scenario taxonomy with tags, and finalize LICENSE
  \end{itemize}
  \note[item]{Cornerstone content repository that needs structural stabilization and metadata standardization.}
\end{frame}

% === DOCUMENTATION & GOVERNANCE ===
\section{Documentation \& Governance}

\begin{frame}{.github – Community Health Files \; \faUsers}
  \textbf{Status:} Active \hfill \textbf{Commits:} 13 \hfill \textbf{Lang:} Markdown/YAML\\[2mm]
  \textbf{Purpose:} Organization-wide default community health files to standardize CODE\_OF\_CONDUCT, CONTRIBUTING, SECURITY, issue/PR templates across public repositories.\\[2mm]
  \textbf{Key Findings}
  \begin{itemize}
    \item Centralizes community health files for all public repos
    \item Add CI to lint templates and validate YAML
    \item Document private-repo strategy and license requirements
  \end{itemize}
  \begin{tcolorbox}
    \faExclamationTriangle\; \textbf{Critical Limitation}: Defaults apply \emph{only} to public repos; private repos need local copies. LICENSE files cannot be inherited.
  \end{tcolorbox}
  \note[item]{Ensures consistent governance across Range42 public projects.}
  \note[item]{LICENSE files are per-repo and private repos won't inherit defaults.}
\end{frame}

\begin{frame}{range42-documentation-private-obsidian \; \faBook}
  \textbf{Status:} Active \hfill \textbf{Commits:} 3 \hfill \textbf{Lang:} Markdown/Shell\\[2mm]
  \textbf{Purpose:} Private Obsidian vault containing architecture canvases, meeting notes, API drafts, and internal documentation.\\[2mm]
  \textbf{Key Findings}
  \begin{itemize}
    \item Private Obsidian vault with architecture canvases and internal documentation; 3 commits only
    \item Contains valuable schemas and meeting notes but lacks public documentation export pipeline
    \item Create automated export to public docs, version control architecture canvases, add pruning cadence
  \end{itemize}
  \note[item]{Knowledge management repository that needs regular maintenance cadence.}
  \note[item]{Consider diagrams-as-code for architecture to enable version control and reviews.}
\end{frame}

% === RISKS & GOVERNANCE ===
\section{Risks \& Governance}

\begin{frame}{Key Risks \& Gaps \; \faExclamationTriangle}
  \textbf{Governance \& Quality}
  \begin{itemize}
    \item \alert{LICENSE placeholders in 9/13 repos}: Replace \texttt{<year>} and \texttt{<name of author>} with actual values
    \item \alert{No SECURITY.md in any repo}: Define vulnerability disclosure and triage process
    \item \alert{No CI/CD in 12/13 repos}: Enable lint, tests, security scans, release checks
    \item \alert{Missing contributor docs}: Add CONTRIBUTING.md, .editorconfig standards
  \end{itemize}
  \vspace{2mm}
  \textbf{Technical Debt}
  \begin{itemize}
    \item API contract formalization needed between backend and UI
    \item Scenario taxonomy and compatibility matrix documentation
    \item Stale repositories need archive-or-activate decisions
  \end{itemize}
  \note[item]{Make CI the gate: ansible-lint, molecule, pytest, npm audit, bandit, safety/pip-audit.}
\end{frame}

% === WRAP-UP ===
\begin{frame}{Takeaways \; \faCheck}
  \begin{itemize}
    \item Range42 = modular, reproducible cyber range for realistic training
    \item \alert{Strong security baseline}: Zero high-severity findings across all code
    \item \alert{Governance is the multiplier}: LICENSE, SECURITY, CI/CD, and contributor docs
    \item Focus areas: complete governance files, enable CI, formalize API contracts
  \end{itemize}
  \vspace{3mm}
  \begin{tcolorbox}
    \faLightbulb\; \textbf{Current Priority}: Fix LICENSE placeholders, add SECURITY.md, enable CI across all active repos.
  \end{tcolorbox}
  \note[item]{Invite questions; emphasize that security is solid, governance unlocks collaboration.}
\end{frame}

\end{document}